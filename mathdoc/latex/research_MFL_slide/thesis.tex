\documentclass[12pt,showkeys]{amsart}
\usepackage[dvipdfmx]{graphicx}
\usepackage{amsfonts}
\usepackage{tikz-cd}
\usepackage{amsmath}
\usepackage[mathscr]{eucal}
\usepackage{amssymb}
\usepackage{latexsym}
\usepackage{amsthm}
\usepackage{amscd}
\usepackage{comment}
\usepackage{enumitem}% http://ctan.org/pkg/enumitem
\usepackage{amsthm}
\usepackage[top=40truemm,bottom=30truemm,left=25truemm,right=25truemm]{geometry}

\theoremstyle{theorem}
\newtheorem{theorem}{Theorem}[section]
\newtheorem{prop}[theorem]{Proposition}
\newtheorem{corollary}[theorem]{Corollary}
\newtheorem{lem}[theorem]{Lemma}
\newtheorem{introtheorem}{Theorem}
\renewcommand{\theintrotheorem}{\Alph{introtheorem}}
\theoremstyle{definition}
\newtheorem{definition}[theorem]{Definition}
\newtheorem{remark}[theorem]{Remark}
\newtheorem{Proof}[theorem]{Proof}

\newenvironment{acknowledgements}%
	{
    \begin{center}%
	\scshape Acknowledgments
    \end{center}\vspace{6pt}
    }%

\def\bN{{\mathbb N}}
\def\bZ{{\mathbb Z}}
\def\bQ{{\mathbb Q}}
\def\bR{{\mathbb R}}
\def\bC{{\mathbb C}}
\def\bF{{\mathbb F}}
\def\bK{{\mathbb K}}

\def\frkg{{\mathfrak g}}
\def\frkh{{\mathfrak h}}
\def\frkk{{\mathfrak k}}
\def\frkr{{\mathfrak r}}
\def\frks{{\mathfrak s}}

\def\frkgl{{\mathfrak {gl}}}
\def\frksl{{\mathfrak {sl}}}
\def\frksp{{\mathfrak {sp}}}
\def\frkso{{\mathfrak {so}}}

\def\scrL{{\mathscr L}}
\def\scrF{{\mathscr F}}

\def\bQbar{{\overline{\mathbb Q}}}

\def\Xbar{{\overline{X}}}
\def\Zp{{{\mathbb Z}_p}}
\def\Qp{{{\mathbb Q}_p}}

\def\an{{\mathrm{an}}}
\def\alg{{\mathrm{alg}}}
\def\op{{\mathrm{op}}}
\def\Gr{{\mathrm{Gr}}}

\def\CalB{{\mathcal B}}
\def\CalF{{\mathcal F}}
\def\CalM{{\mathcal M}}
\def\CalO{{\mathcal O}}
\def\CalI{{\mathcal I}}

\def\peh{{{\mathcal P}}}

\def\profin{{\widehat{~}}}
\def\prol{{(\ell)}}

\def\id{{\mbox{id}}}
\def\Ker{{\mbox{Ker}}}
\def\Im{{\mbox{Im}}}
\def\mod{{\mbox{ mod }}}
\def\Hom{{\mbox{Hom}}}
\def\End{{\mbox{End}}}
\def\Aut{{\mbox{Aut}}}
\def\Inn{{\mbox{Inn}}}
\def\Out{{\mbox{Out}}}
\def\Gal{{\mbox{Gal}}}
\def\Lie{{\mbox{Lie}}}
\def\Der{{\mbox{Der}}}

\def\defeq{ \ {\stackrel{\mathrm{def}}{=}} \ }


\title[On the outer automorphism groups]{On the outer automorphism groups of the absolute Galois groups of mixed-characteristic local fields}
\author{Yuichiro Hoshi \and Yu Nishio \\ \\ November 2020}
\subjclass[2010]{11S20}
\keywords{mixed-characteristic local field, absolute Galois group, 
anabelian geometry, mono-anabelian geometry, 
group of MLF-type}
\date{}

\begin{document}

\maketitle

\begin{abstract}
In the present paper, we study the outer automorphism groups of the 
absolute Galois groups of mixed-characteristic local fields from the 
point of view of anabelian geometry.  Let us recall that it is 
well-known that the natural homomorphism from the automorphism group 
of a mixed-characteristic local field to the outer automorphism 
group of the absolute Galois group of the given mixed-characteristic 
local field is injective.  One main result of the present paper is 
that if the mixed-characteristic local field satisfies certain 
conditions, then the set of conjugates of the image of this injective 
homomorphism in the outer automorphism group is infinite, which thus 
implies that the image of this injective homomorphism is not normal in 
the outer automorphism group. 
In particular, one may conclude that it is impossible to establish a functorial group-theoretic reconstruction, from the absolute Galois group, 
of the ``field-theoretic'' subgroup, i.e., the image of this injective 
homomorphism, of the outer automorphism group.  
\end{abstract} 

\setcounter{section}{-1}

\section*{Introduction}
Let $p$ be a prime number, $k$ a finite extension of 
$\mathbb{Q}_p$, and $\overline{k}$ an algebraic closure 
of $k$.  Write $G_k \stackrel{\mathrm{def}}{=} \mathrm{Gal}
(\overline{k} / k)$ for the absolute Galois group of $k$ 
determined by the algebraic closure $\overline{k}$ 
and $\mathrm{Out}(G_k)$ for the group of outer 
automorphisms of the group $G_k$ [or, alternatively, the 
group of outer continuous automorphisms of the profinite 
group $G_k$ --- cf., e.g., \cite{Hoshi4}, Proposition 1.2, 
(i), (ii)].  In the present paper, we study the outer automorphism group 
$\mathrm{Out}(G_k)$ from the point of view of anabelian 
geometry.  


Write $\mathrm{Aut}(k)$ for the group of automorphisms of 
the field $k$.  Thus, we have a natural homomorphism 
$\mathrm{Aut}(k) \to \mathrm{Out}(G_k)$ of groups.  Let us 
first recall that it is well-known [cf., e.g., \cite{Hoshi1}, 
Proposition 2.1] that this homomorphism is injective.  In 
the present paper, let us regard $\mathrm{Aut}(k)$ as a 
[necessarily finite] subgroup of $\mathrm{Out}(G_k)$ by 
means of this injective homomorphism:  
\[
\mathrm{Aut}(k) \subseteq \mathrm{Out}(G_k).  
\]
Here, we note that it is well-known [cf., e.g., the 
discussion given at the final portion of \cite{NSW}, 
Chapter VII, \S 5] that [although a similar equality always holds for a finite extension 
of $\mathbb{Q}$ by the Neukirch-Uchida theorem --- cf., e.g., 
\cite{NSW}, Corollary 12.2.2], in general, the equality 
$\mathrm{Aut}(k) = \mathrm{Out}(G_k)$ does not hold.  
In particular, one may conclude that, roughly speaking, 
in general, a finite extension of $\mathbb{Q}_p$ 
should be considered to be ``not anabelian'' [cf.\ also \cite{NSW}, Chapter XII, \S 2, Closing remark]. Therefore, one main interest from the point of view of anabelian 
geometry is in the investigate of the ``difference'' between 
$\mathrm{Aut}(k)$ and $\mathrm{Out}(G_k)$.  
Some results concerning the ``characterization'' of the subgroup 
$\mathrm{Aut}(k)$ of $\mathrm{Out}(G_k)$ may be found in 
\cite{Mzk}, \S 3, and \cite{Hoshi-b}, 
\S 3.  Moreover, some results concerning this  ``difference'' may be found in \cite{Hoshi2}, \S 7, and 
\cite{Hoshi2}, \S 8.    


Write $(\mathbb{Q}_p)_+ \subseteq k_+$ for the underlying 
additive modules of the fields $\mathbb{Q}_p \subseteq k$, 
respectively.  Next, let us recall that, by applying a 
functorial group-theoretic reconstruction algorithm 
established in the study of the mono-anabelian geometry of mixed-characteristic local fields [cf., 
e.g., \cite{Hoshi1}, Definition 3.10, (vi), and \cite{Hoshi1}, 
Proposition 3.11, (iv)], one obtains an action of the group 
$\mathrm{Out}(G_k)$ on the module $k_+$ whose restriction 
to the above subgroup $\mathrm{Aut}(k) \subseteq 
\mathrm{Out}(G_k)$ coincides with the natural action of 
$\mathrm{Aut}(k)$ on $k_+$.  


One main technical result of the present paper is as 
follows [cf.\ Theorem \ref{one:theorem:2.8}]:

\begin{introtheorem}
\label{AAAAAAA}
Suppose that the following three conditions are 
satisfied:  
\begin{itemize}
\item[$(1)$]
The prime number $p$ is odd.  
\item[$(2)$]
The finite extension $k / \mathbb{Q}_p$ is 
of even degree.  
\item[$(3)$]
The finite extension $k / \mathbb{Q}_p$ is Galois, 
and, moreover, the Galois group 
$\mathrm{Gal}(k / \mathbb{Q}_p)$ is abelian.  
\end{itemize}
Then there exists an outer automorphism $\alpha$ of $G_k$ 
such that, for each nonzero integer $n$, if one writes 
$\alpha_+^n$ for the action of $\alpha^n$ on $k_+$, then 
$\alpha^n_+((\mathbb{Q}_p)_+) \neq (\mathbb{Q}_p)_+$.  
\end{introtheorem}

Next, let us recall that the first author of the 
present paper proved that if $p$ is odd, and $k$ 
coincides with the [necessarily finite Galois] 
extension of $\mathbb{Q}_p$ obtained by adjoining 
a primitive $p$-th root of unity and a $p$-th root 
of $p \in \mathbb{Q}_p$, then the above subgroup 
$\mathrm{Aut}(k) \subseteq \mathrm{Out}(G_k)$ is 
not normal [cf.\ \cite{Hoshi2}, Theorem G, (iii)].  
In the present paper, we give a proof of an assertion 
in this direction by applying Theorem~\ref{AAAAAAA}.  
More precisely, in the present paper, we prove the 
following result [cf.\ Theorem \ref{two:theorem:2.8}]:  


\begin{introtheorem}
\label{BBBBBBB}
Suppose that the three conditions in the statement of 
Theorem~\ref{AAAAAAA} are satisfied.  Then the set of 
$\mathrm{Out}(G_k)$-conjugates of the subgroup 
$\mathrm{Aut}(k) \subseteq \mathrm{Out}(G_k)$ is 
infinite.  
\end{introtheorem}


A formal consequence of Theorem~\ref{BBBBBBB} is as 
follows [cf.\ Corollary \ref{corollary:2.9}]:  


\begin{introtheorem}
\label{CCCCCCC}
Suppose that the three conditions in the statement of 
Theorem~\ref{AAAAAAA} are satisfied.  
Then the following hold:  
\begin{itemize}
\item[$(\mathrm{i})$]
The subgroup $\mathrm{Aut}(k) \subseteq 
\mathrm{Out}(G_k)$ is not normal.  
\item[$(\mathrm{ii})$]
There exist infinitely many distinct [necessarily 
finite] subgroups of $\mathrm{Out}(G_k)$ isomorphic 
to $\mathrm{Aut}(k)$.  
\end{itemize}
\end{introtheorem}


The issue of whether or not a functorial group-theoretic 
reconstruction, from the group $G_k$, of the 
``field-theoretic'' subgroup 
$\mathrm{Aut}(k) \subseteq \mathrm{Out}(G_k)$ 
of the outer automorphism group $\mathrm{Out}(G_k)$ 
can be established is interesting from the 
point of view of the anabelian geometry of 
mixed-characteristic local fields.  
Now let us recall that if the condition (3) in the 
statement of Theorem~\ref{AAAAAAA} 
is satisfied, then, roughly speaking, 
one may reconstruct group-theoretically, 
from the group $G_k$, the set of 
$\mathrm{Out}(G_k)$-conjugates of the 
subgroup $\mathrm{Aut}(k) \subseteq \mathrm{Out}(G_k)$ 
[cf.\ \cite{Hoshi2}, Theorem F, (i), and 
\cite{Hoshi2}, Theorem 6.12, (ii)].  
On the other hand, Theorem~\ref{CCCCCCC}, (i), 
implies that if the three conditions in the statement 
of Theorem~\ref{AAAAAAA} are satisfied, then, 
roughly speaking, it is impossible to establish 
a functorial group-theoretic reconstruction of the 
subgroup $\mathrm{Aut}(k) \subseteq \mathrm{Out}(G_k)$ 
itself [i.e., as opposed to the set of $\mathrm{Out}(G_k)$-conjugates of the subgroup 
$\mathrm{Aut}(k) \subseteq \mathrm{Out}(G_k)$].  

\section{Notational conventions}\label{section0}

\subsubsection*{\sc Sets} 
 
If $G$ is a group, and $T$ is a set equipped with an action of $G$, then we shall write $T^G \subseteq T$ for the subset of $G$-invariants of $T$.

\subsubsection*{\sc Topological groups} 
 
If $G$ is a topological group, then we shall write $G^{\rm ab}$ for the abelianization of $G$ [i.e., the quotient of $G$ by the closure of the commutator subgroup
of $G$] and $G^{\rm ab/tor}$ for the quotient of $G^{\rm ab}$ by the closure of the subgroup of $G^{\mathrm{ab}}$ of torsion elements. 

\subsubsection*{\sc Rings}
 
If $R$ is a ring, then we shall write $R_+$ for the underlying additive module of $R$ and $R^\times$ for the multiplicative module of units of $R$.  

\subsubsection*{\sc Fields} 
 
We shall refer to a field isomorphic to a finite extension of $\Qp$, for some prime number $p$, as an $MLF$. Here, ``MLF'' is to be understood as an abbreviation for ``mixed-characteristic local field''. 

\section{Existence of an automorphism with a certain unipotency condition of a group of MLF-type}\label{section1}

In the present \S \ref{section1}, we prove that a certain group of MLF-type admits an automorphism that
satisfies a certain unipotency condition [cf.\ Theorem~\ref{theorem:1.5}
below].
In the present \S \ref{section1}, let $G$ be a [profinite --- cf.\ \cite{Hoshi4}, Proposition 1.2, (i), (ii)] group of MLF-type [cf.\ \cite{Hoshi4}, Definition 1.1]. Thus, by applying the various functorial group-theoretic reconstruction algorithms of \cite{Hoshi1}, \S 3 [cf.\ \cite{Hoshi1}, Definition 3.5, (i), (ii), (iii); \cite{Hoshi1}, Definition 3.10, (ii), (iv), (vi)], to  the group $G$ of MLF-type, we obtain
\begin{itemize}
  \item a prime number $p(G)$,
  \item positive integers $d(G)$ and $f(G)$, 
  \item a normal closed subgroup $P(G) \subseteq G$ of $G$, and
  \item topological modules $\CalO^{\prec}(G) \subseteq k^\times (G)$ and $k_+(G)$ 
\end{itemize}
[cf.\ also \cite{Hoshi1}, Summary 3.15]. 
In the present \S \ref{section1}, suppose, moreover, that $p(G)$ is odd, and that $d(G) > 1$. 

\begin{prop}[Jannsen-Wingberg]\label{proposition:1.1} 
The profinite group $G$ is topologically generated by $d(G)+3$ elements $\sigma$, $\tau$, $x_0, \dots, x_{d(G)}$ subject to the following conditions and relations.

\begin{enumerate}[label=(\roman*),ref=(\roman*)]
  \item[\rm (1)] The normal closed subgroup $P(G)$ of $G$ is topologically normally generated by $x_0, \dots, x_{d(G)}$.
  \item[\rm (2)] The elements $\sigma$, $\tau$ satisfy the relation $\sigma \tau \sigma^{-1} = \tau^{p(G)^{f(G)}}$.
  \item[\rm (3)] The topological generators under consideration satisfy the relation 
\[
\sigma x_0 \sigma^{-1}
=
(x^{h^{{p(G)}-1}}_0 \tau x^{h^{{p(G)}-2}}_0 \tau \cdots x^{h}_0 \tau)^{\frac{\pi g}{{p(G)}-1}}
x_1^{p(G)^{s}}
\delta ,
\]
where $s, g, h$ are some positive integers such that 
\[
g(h^{p(G) - 1} + h^{p(G) - 2} + \cdots + h) \neq p(G) - 1, 
\]
$\pi$ is the unique element of $\hat{\bZ} = \prod_p \bZ_p$ whose image in $\bZ_p$ is given by $1$ if $p = p(G)$ $($resp.\ by $0$ if $p\neq p(G)$$)$, and $\delta$ is an element of the commutator subgroup of $G$.  
\end{enumerate}
\end{prop}
\begin{proof}
This assertion follows from \cite{NSW}, Theorem 7.5.14, together with \cite{Hoshi1}, Proposition 3.6. [Note that it follows from the discussion preceding \cite{NSW}, Theorem 7.5.14, that one may take the ``$g$'' and ``$h$'' of \cite{NSW}, Theorem 7.5.14, to be positive integers greater than $p(G)$.]  
\end{proof}

In the remainder of the present \S \ref{section1}, let us fix 
topological generators $\sigma$, $\tau$, $x_0, \dots, x_{d(G)}$ of 
$G$ as in Proposition \ref{proposition:1.1}. Write $S \defeq \{0, 
1, \dots, d(G)\}$. 
Moreover, for each $i \in S$, write 
\begin{itemize}
\item
$y_i \in k_+(G)$ for the image of $x_i$ in $k_+(G)$ 
and 
\item
$z_i \in \CalO^{\prec}(G)^{\rm ab/tor}$ for the image 
of $x_i$ in $\CalO^{\prec}(G)^{\rm ab/tor}$ 
\end{itemize}
[cf.\ the condition (1) of Proposition \ref{proposition:1.1}; 
\cite{Hoshi1}, Lemma 1.5, (ii); \cite{Hoshi1}, Proposition 3.6; 
\cite{Hoshi1}, Definition 3.10, (i), (ii), (vi)]. 

\begin{lem}\label{lemma:1.2} 
 The topological module $k_{+}(G)$ has a natural structure of ${\bQ}_{p(G)}$-vector space of dimension $d(G)$. 
\end{lem}

\begin{proof}
It follows immediately from the definition of $k_{+}(G)$ [cf.\ \cite{Hoshi1}, Definition 3.10, (vi)] that $k_{+}(G)$ has a natural structure of ${\bQ}_{p(G)}$-vector space. Moreover, it follows from \cite{Hoshi1}, Proposition 3.6, and  \cite{Hoshi1}, Proposition 3.11, (iv), that this $\bQ_{p(G)}$-vector space $k_{+}(G)$ is of dimension $d(G)$. 
\end{proof}

\begin{lem}\label{lemma:1.3} 
The $d(G)$ elements $y_1, \dots, y_{d(G)}$ form a basis of the $\bQ_{p(G)}$-vector space $k_{+}(G)$ [cf.\ Lemma \ref{lemma:1.2}]. 
\end{lem}

\begin{proof}
Since ${\{x_i\} }_{i \in S}$ topologically normally generates $P(G)$ [cf.\ the condition (1) of Proposition \ref{proposition:1.1}], ${\{z_i\} }_{i \in S}$ topologically generates $\CalO^{\prec}(G)^{\rm ab/tor}$ $( \subseteq G^{\rm ab/tor} )$ [cf.\ \cite{Hoshi1}, Definition 3.10, (i), (ii)]. Moreover, since $\CalO^{\prec}(G)^{\rm ab/tor} \otimes_{\mathbb{Z}_{p(G)}} (\mathbb{Z}_{p(G)} / p(G)^n\mathbb{Z}_{p(G)})$ is a finite $p(G)$-group for every positive integer $n$ [cf.\ \cite{Hoshi1}, Proposition 3.11, (i)], ${\{z_i\} }_{i \in S}$ is also a generator of $\CalO^{\prec}(G)^{\rm ab/tor}$ even if we regard $\CalO^{\prec}(G)^{\rm ab/tor}$ as a $\bZ_{p(G)}$-module. Next, let us observe that it follows from the relation (2) of Proposition \ref{proposition:1.1} that the image of $\tau$ in $G^{\rm ab/tor}$ is trivial. Thus, it follows from the relation (3) of Proposition \ref{proposition:1.1} that the relation $1={z_0}^{H}{z_1}^{{p(G)}^s}$ in $\CalO^{\prec}(G)^{\rm ab/tor}$ holds for some nonzero [cf.\ the second display of Proposition \ref{proposition:1.1}, (3)] integer $H$. Therefore, if we write $T \defeq S \backslash \{0\}$, then ${\{z_i \otimes 1 \} }_{i \in T}$ is a generator of $\bQ_{p(G)}$-vector space $\CalO^{\prec}(G)^{\rm ab/tor} \otimes_{\bZ_{p(G)}}  \bQ_{p(G)}$. Here, let us observe that we have a natural topological isomorphism $k_{+}(G) \stackrel{\sim}{\to} \CalO^{\prec}(G)^{\rm ab/tor} \otimes_{\bZ_{p(G)}}  \bQ_{p(G)}$, by definition, that maps $y_i \in k_{+}(G)$ to $z_i \otimes 1 \in  \CalO^{\prec}(G)^{\rm ab/tor} \otimes_{\bZ_{p(G)}} \bQ_{p(G)}$ for each $i \in S$. This isomorphism is also an isomorphism of $\bQ_{p(G)}$-vector spaces by construction. Moreover, since $k_{+}(G) \simeq \CalO^{\prec}(G)^{\rm ab/tor} \otimes_{\bZ_{p(G)}}  \bQ_{p(G)}$ is a $\bQ_{p(G)}$-vector space of dimension $d(G)$ [cf.\ Lemma \ref{lemma:1.2}], ${\{{z_i} \otimes 1 \} }_{i \in T}$ is a basis of the $\bQ_{p(G)}$-vector space $\CalO^{\prec}(G)^{\rm ab/tor} \otimes_{\bZ_{p(G)}}  \bQ_{p(G)}$. Therefore, it follows from the condition imposed on the isomorphism $k_{+}(G) \stackrel{\sim}{\to} \CalO^{\prec}(G)^{\rm ab/tor} \otimes_{\bZ_{p(G)}}  \bQ_{p(G)}$ that ${\{y_i\} }_{i \in T}$ is a basis of the $\bQ_{p(G)}$-vector space $k_{+}(G)$. 
\end{proof}

\begin{lem}\label{lemma:1.4} 
The element $y_{d(G)-1}$ is a $nonzero$ element of $k_{+}(G)$. 
\end{lem}
\begin{proof} 
This assertion follows from Lemma \ref{lemma:1.3} [cf.\ also our assumption that $d(G) > 1$]. 
\begin{comment}
From the relation (\ref{lemma:1.3:eq:1}) in the proof of Lemma \ref{lemma:1.3}, $y_0$ and $y_1$ are linearly dependent. Suppose that $y_{d-1} = 0$. Since $y_0$ and $y_1$ are linearly dependent and $y_{d-1} = 0$, the dimension of the subspace spanned by $y_0, \dots, y_d$ is less than $(d+1)-2 = d-1$. From Lemma \ref{lemma:1.3} and the previous discussion, the dimension of $k_+(G)$ is less than $d-1$. This contradicts Lemma \ref{lemma:1.2}. 
First, we assume that $y_{d(G)-1} = 0$ and $d(G) = 2$. Since $\Ker(\CalO^{\times}(G) \rightarrow k_{+}(G)) = \CalO^{\prec 1}(G)_{\rm tor}$ [cf. \cite{Hoshi1}, Lemma 1.2, (v)], we conclude from the assumption that $y_{d(G) - 1} = 0$ that $x_{d(G)-1} \in \CalO^{\prec 1}(G)_{\rm tor}$. Thus, $z_{d(G)-1} = z_1$ is the identity element. In paticular, it follows from the condition of ${\{z_i\} }_{i \in S}$ [cf. Lemma \ref{lemma:1.3}] that $z_0$ is the identity element. Thus, $y_2$ is the only nontrivial element of ${\{y_i\} }_{i \in S}$. However this contradicts the fact that $k_{+}(G)$ is a $\bQ_{p(G)}$-vector space of dimension 2 [cf. \cite{Hoshi1}, Proposition 3.11, (iv)]. 
Next, we assume that $y_{d(G)-1} = 0$ and $d(G) > 2$. Since $y_0$ and $y_1$ are linearly dependent [cf. the relation (\ref{lemma:1.3:eq:1}) in the proof of Lemma \ref{lemma:1.3}], and $y_{d(G)-1} = 0$, the number of linearly independent elements of ${\{y_i\} }_{i \in S}$ is less than $d(G)-1$. It contradicts the fact that $k_{+}(G)$ is a $\bQ_{p(G)}$-vector space of dimension $d(G)$ [cf. \cite{Hoshi1}, Proposition 3.11, (iv)]. Since $y_{d(G)-1} = 0$, together with Lemma \ref{lemma:1.3}, ${\{y_i\} }_{i \in S \backslash (d(G)-1)}$ is a basis of the $\bQ_{p(G)}$-vector space $k_{+}(G)$. However, since $y_0$ and $y_1$ are linearly dependent [cf. the relation (\ref{lemma:1.3:eq:1}) in the proof of Lemma \ref{lemma:1.3}], ${\{y_i\} }_{i \in S \backslash (d(G)-1)}$ is not a basis of the $\bQ_{p(G)}$-vector space $k_{+}(G)$. This is cleary a contradiction.  
\end{comment}
\end{proof}

\begin{theorem}\label{theorem:1.5}
 Let $G$ be a group of MLF-type.  Suppose that $p(G)$ is odd, and that $d(G) > 1$.  Then there exists an automorphism $\alpha$ of $G$ such that, for each nonzero integer $n$, if one writes $\alpha^n_+$ for the automorphism of the $\mathbb{Q}_{p(G)}$-vector space $k_+(G)$ induced by $\alpha^n$, then $\alpha^n_+ \neq \mathrm{id}$, and, moreover, the equality $(\alpha^n_{+}- {\rm id})^2 = 0$ in the ring of endomorphisms of $k_+(G)$ holds.  
\begin{proof} 
Let $\alpha$ be an automorphism of $G$ as in the discussion preceding \cite{NSW}, Theorem 7.5.15, i.e., defined by the equalities $\alpha(\sigma) = \sigma$, $\alpha(\tau) = \tau$, $\alpha(x_{d(G)}) = x_{d(G)} x_{d(G)-1}$, and $\alpha(x_i) = x_i$ for $i\in S \backslash \{d(G)\}$. First, we prove that $\alpha_+^n \neq \mathrm{id}$ for each nonzero integer $n$. If $\alpha_+^n = {\rm id}$, then $y_{d(G)} = \alpha^n_+(y_{d(G)}) = y_{d(G)} + ny_{d(G)-1}$, which thus implies that $y_{d(G)-1} = 0$ in $k_{+}(G)$. However, this contradicts Lemma \ref{lemma:1.4}. Thus, we conclude that $\alpha_+^n \neq {\rm id}$. Next, let us observe that, for each nonzero integer $n$, it follows from the easily verified equality $(\alpha^n_+ - {\rm id})^2 (y_i) = 0$ for every $i \in S$ and Lemma \ref{lemma:1.3} that the equality $(\alpha^n_+- {\rm id})^2 = 0$ in $\End(k_+(G))$ holds, as desired. This completes the proof of Theorem \ref{theorem:1.5}. 
\end{proof}
\end{theorem}

\begin{corollary}\label{corollary:2.12}
Let $G$ be a group of MLF-type.  
Suppose that $p(G)$ is odd, and that $d(G) > 1$.  
Then the following hold: 
\begin{enumerate}[label=(\roman*),ref=(\roman*)]
  \item[\rm (i)] The image of the natural homomorphism from the outer automorphism 
group of $G$ to the automorphism group of $k_+(G)$ is infinite. \label{corollary:2.12:statement:1}
  \item[\rm (ii)] The image of the natural homomorphism from the outer automorphism 
group of $G$ to the automorphism group of $G^{\mathrm{ab}}$ is infinite. \label{corollary:2.12:statement:2}
  \item[\rm (iii)] The image of the natural homomorphism from the outer automorphism 
group of $G$ to the automorphism group of $k^\times(G)$ is infinite. \label{corollary:2.12:statement:3}
  \item[\rm (iv)] The outer automorphism group of $G$ is infinite. \label{corollary:2.12:statement:4}
\end{enumerate}
\end{corollary}
\begin{proof}
Assertion (i) follows from Theorem~\ref{theorem:1.5}.  
Assertion (ii) follows from assertion (i), 
together with the definition of $k_+(G)$ 
[cf.\ \cite{Hoshi1}, Definition 3.10, (vi)].  
Assertion (iii) follows from assertion (ii) and 
the [easily verified] density of $k^\times(G)$ in $G^{\mathrm{ab}}$ 
[cf.\ \cite{Hoshi1}, Definition 3.10, (iv)].  
Assertion (iv) follows from assertion (i).  
\end{proof}

\begin{remark}\label{remark:2.13}
Let us recall that it follows immediately from 
\cite{Hoshi2}, Corollary 5.5, that each of the three images 
discussed in Corollary \ref{corollary:2.12}, 
(i), (ii), (iii), in the case where $d(G)$ is
equal to $1$ is trivial.
\end{remark}

\section{Existence of a special automorphism of the absolute Galois group of an absolutely abelian MLF of even degree}\label{section2}

In the present \S\ref{section2}, we prove that the absolute Galois group of a certain MLF admits
an automorphism that has an interesting property [cf.\ Theorem \ref{one:theorem:2.8} below]. In  the present \S \ref{section2}, let $k$ be an MLF and $\overline{k}$ an algebraic closure of $k$. 
We shall write
\begin{itemize}
  \item $G_k \defeq \mathrm{Gal}(\overline{k}/k)$ for the absolute Galois group of $k$ determined by the algebraic closure $\overline{k}$, 
  \item $\CalO_k$ for the ring of integers of $k$,
  \item $p_k$ for the residue characteristic of $k$,
  \item $k^{(d=1)} \subseteq k$ for the [uniquely determined] minimal MLF contained in $k$,
  \item $d_k \defeq [k:k^{(d=1)}]$ for the degree of the finite extension $k/k^{(d=1)}$, 
  \item ${\rm Nm}_{k/k^{(d=1)}}\colon k^\times \to {k^{(d = 1)}}^\times $ for the norm map with respect to the finite extension $k/k^{(d=1)}$, 
  \item ${\rm Tr}_{k/k^{(d=1)}}\colon k_+ \to k^{(d = 1)}_+ $ for the trace map with respect to the finite extension $k/k^{(d=1)}$,  
  \item ${\rm log}_k \colon \CalO^{\times}_k \rightarrow k_+$ for the $p_k$-adic logarithm, and 
  \item $\mathcal{I}_k$ for the log-shell of $k$. 
\end{itemize}
Write, moreover, $\mathrm{Aut}(G_k)$, $\mathrm{Aut}(k_+)$, and $\mathrm{Aut}(k^\times)$ for the groups of automorphisms of the group $G_k$, the module $k_+$, and the module $k^\times$, respectively.  Thus, it follows from \cite{Hoshi1}, Proposition 3.11, (i), (iv), that we have homomorphisms 
\begin{equation}
\Aut(G_k) \longrightarrow \Aut(k_+), \ \Aut(G_k) \longrightarrow \Aut(k^\times). \nonumber
\end{equation}

\begin{definition}\label{definition:2.1}    
       \ \ \ 
\begin{enumerate}[label=(\roman*),ref=(\roman*)]
   \item[\rm (i)]  We shall say that $\alpha \in \Aut(k_+)$ is $(\bQ_{p_k})_+$-{\it characteristic} if $\alpha(k^{(d=1)}_+) = k^{(d=1)}_+$. \label{definition:2.1:statement1} 
   \item[\rm (ii)] We shall say that $\alpha \in \Aut(k_+)$ is $(\bQ_{p_k})_+$-{\it preserving} if $\alpha$ is $(\bQ_{p_k})_+$-characteristic, and $\left. \alpha \right|_{k^{(d=1)}_+}$ is the identity automorphism of $k_+^{(d=1)}$.  \label{definition:2.1:statement:2} 
   \item[\rm (iii)]  We shall say that $\alpha \in \Aut(k_+)$ is {\it group-theoretic} if $\alpha$ is contained in the image of the first homomorphism $\Aut(G_k) \to \Aut(k_+)$ of the above display. \label{definition:2.1:statement:3} 
\item[\rm (iv)]  We shall say that $\alpha \in \Aut(k^{\times})$ is {\it group-theoretic} if $\alpha$ is contained in the image of the second homomorphism $\Aut(G_k) \to \Aut(k^\times)$ of the above display. \label{definition:2.1:statement:4}    
   \item[\rm (v)]  We shall say that $\alpha \in \Aut(G_k)$ is $(\bQ_{p_k})_+$-{\it characteristic} if the group-theoretic automorphism of $k_+$ induced by $\alpha$ is $(\bQ_{p_k})_+$-characteristic. \label{definition:2.1:statement:5} 
   \item[\rm (vi)] We shall say that $\alpha \in \Aut(G_k)$ is $(\bQ_{p_k})_+$-{\it preserving} if the group-theoretic automorphism of $k_+$ induced by $\alpha$ is $(\bQ_{p_k})_+$-preserving.  \label{definition:2.1:statement:6} 
\end{enumerate}
\end{definition}

\begin{lem}\label{lemma:new:2.2}
The diagram of modules 
\[
\begin{tikzcd}
\mathcal{O}_k^{\times} \arrow[r, "{\log}_k"] \arrow[d, "{\rm Nm}_{k/k^{(d=1)}}"'] & k_+ \arrow[d, "{\rm Tr}_{k/k^{(d=1)}}"] \\
\mathcal{O}_{k^{(d=1)}}^{\times} \arrow[r, "{\log}_{k^{(d=1)}}"']                              & k^{(d=1)}_+             
\end{tikzcd}
\]
commutes.
\end{lem}
\begin{proof}
Since $k^{(d=1)}_+$ is torsion-free, by replacing $k$ by the Galois closure of $k$ over $k^{(d=1)}$, we may assume without loss of generality that $k$ is absolutely Galois, i.e., that $k$ is Galois over $k^{(d=1)}$ [cf.\ \cite{Hoshi2}, Definition 4.2, (i)].  
Then Lemma \ref{lemma:new:2.2} follows immediately from the well-known fact that the $p_k$-adic logarithm is compatible with the respective natural actions of ${\rm Gal} (k/ k^{(d=1)})$ on $\mathcal{O}_k^\times$ and on $k_+$.  
\end{proof}


\begin{lem}\label{lemma:2.2}
Let $\alpha$ be an automorphism of $G_k$.  Write $\alpha_+ \in \mathrm{Aut}(k_+)$ and $\alpha^{\times} \in \mathrm{Aut}(k^{\times})$ for the respective group-theoretic automorphisms induced by $\alpha$. Then the following hold:
\begin{enumerate}[label=(\roman*),ref=(\roman*)]
   \item[\rm (i)] The automorphism $\alpha^\times$ fits into a commutative diagram of modules
\[
\begin{tikzcd}
k^\times \arrow[rr, "{\rm Nm}_{k/k^{(d=1)}}"] \arrow[d, "\alpha^\times"'] && {k^{(d=1)}}^\times \arrow[d, equal] \\
k^\times \arrow[rr, "{\rm Nm}_{k/k^{(d=1)}}"']                         && {k^{(d=1)}}^\times.                
\end{tikzcd}
\]\label{lemma:2.2:statement:1}
   \item[\rm (ii)] The automorphism $\alpha_{+}$ fits into a commutative diagram of modules
\[
\begin{tikzcd}
k_{+} \arrow[rr, "{\rm Tr}_{k/k^{(d=1)}}"] \arrow[d, "\alpha_{+}"'] && k^{(d=1)}_+ \arrow[d, equal] \\
k_{+} \arrow[rr, "{\rm Tr}_{k/k^{(d=1)}}"']                         && k^{(d=1)}_+.                
\end{tikzcd}
\]
In particular, the automorphism $\alpha_+$ restricts to an  automorphism of \ ${\rm Ker}({\rm Tr}_{k/k^{(d=1)}})$, i.e., the equality $\alpha_+({\rm Ker}({\rm Tr}_{k/k^{(d=1)}})) = {\rm Ker}({\rm Tr}_{k/k^{(d=1)}})$ holds.
\end{enumerate} \label{lemma:2.2:statement:2}
\end{lem}

\begin{proof}
Assertion (i) follows immediately from \cite{Hoshi2}, Proposition 4.9, (i). 
Next, we verify assertion (ii). 
Let us first recall that it follows from  the construction of $\alpha_+$ [cf.\ \cite{Hoshi1}, Definition 3.10, (vi)] and the definition of log-shell that the diagram
\[
\begin{tikzcd}
\mathcal{O}_k^{\times} \arrow[r, "{\rm log}_k"] \arrow[d, "\alpha^{\times}"'] & \CalI_k \arrow[d, "\alpha_{+}"] \\
\mathcal{O}_k^{\times} \arrow[r, "{\rm log}_k"']                              & \CalI_k             
\end{tikzcd}
\]
commutes.
Therefore, by Lemma \ref{lemma:new:2.2} and assertion (i), we get the equality 
\[
{\rm Tr}_{k/k^{(d=1)}}(\alpha_+(\log_k(x))) = {\rm Tr}_{k/k^{(d=1)}}(\log_k(x)) \ \ \ (x \in \mathcal{O}_k^\times).  
\]
Now let us observe that this equality implies that $\alpha_{+}$ is compatible with the trace map with respect to the finite extension $k/k^{(d=1)}$ on $p_k \mathcal{I}_k$. 
Since, for an arbitrary $x \in k_+$, there exists an integer $n$ such that $p_k^{n}x \in p_k \mathcal{I}_k$ [cf.\ \cite{Hoshi1}, Lemma 1.2, (vi)], we conclude that $\alpha_{+} $ is compatible with the trace map on $k_{+}$. 
\end{proof}

\begin{lem}\label{lemma:2.5}
Suppose that $p_k$ is odd, and that $d_k = 2$. Then there exists an automorphism $\alpha \in {\rm Aut}(G_k)$ such that, for every nonzero integer $n$, $\alpha^n$ is not $(\bQ_{p_k})_{+}$-characteristic. 
\end{lem}
\begin{proof}
It follows from Theorem \ref{theorem:1.5} and \cite{Hoshi1}, Proposition 3.6, that there exists a group-theoretic automorphism $\alpha_{+} \in \Aut(k_+)$ such that, for every nonzero integer $n$, $\alpha_+^n$ is not the identity automorphism but satisfies the equality  $(\alpha^n_{+} -{\rm id})^2 = 0$ in $\End(k_+)$. 
Here, let us observe that we can write $k = k^{(d=1)}(\sqrt{a})$ for some $a \in k^{(d=1)}$. 
Assume that $\alpha^n_+$ is $(\bQ_{p_k})_{+}$-characteristic for some nonzero integer $n$. 
Thus, $\alpha^n_+(1) = b$ for some $b \in k^{(d=1)}$.  Moreover, it follows from the final portion of Lemma \ref{lemma:2.2}, (ii), that $\alpha^n_+(\sqrt{a}) = c\sqrt{a}$ for some $c \in k^{(d=1)}$.  Thus, since $\alpha_+$ is an automorphism of $\bQ_{p_k}$-vector space, it follows that, for arbitrary $x$, $y \in k^{(d=1)}$, the equalities 
\begin{eqnarray}
0 = (\alpha^n_{+} -{\rm id})^2(x+y\sqrt{a}) = x (b - 1)^2 + y (c-1)^2 \sqrt{a}  \nonumber
\end{eqnarray}
hold. Thus, we have $(b, c) = (1, 1)$. 
In particular, $\alpha^n_{+}$ is the identity automorphism. However, this is a contradiction. 
\end{proof}

\begin{remark}\label{remark:2.6}
One may conclude from Lemma \ref{lemma:2.5} that 
it is impossible to establish a functorial group-theoretic 
reconstruction algorithm for constructing, from an arbitrary group 
$G$ of MLF-type, a submodule of the module $k_+(G)$ 
which ``corresponds'' to the submodule $k^{(d=1)}_+ 
\subseteq k_+$ of $k_+$.  Put another way, one may 
conclude from Lemma \ref{lemma:2.5} that the 
submodule $k^{(d=1)}_+ \subseteq k_+$ of $k_+$ should 
be considered to be ``not group-theoretic''.  
\end{remark}

\begin{lem}\label{old:lemma:2.7}
Suppose that $d_k$ is even, and that  $k$ is absolutely abelian, i.e., that $k$ is Galois over $k^{(d=1)}$, and, moreover, the Galois group ${\rm Gal}(k/k^{(d=1)})$ is abelian [cf.\ \cite{Hoshi2}, Definition 4.2, (ii)]. Then the following hold: 
\begin{enumerate}[label=(\roman*),ref=(\roman*)]
	\item[\rm (i)] There exists a quadratic extension $k'$ of $k^{(d = 1)}$ contained in $k$  such that $G_k$ is a characteristic subgroup of $G_{k'} \defeq {\rm Gal}(\overline{k}/k')$. In particular, we have a natural homomorphism $\phi \colon {\rm Aut}(G_{k'}) \rightarrow {\rm Aut}(G_k)$. \label{lemma:2.7:statement:1} 
    \item[\rm (ii)] Let $k'$ be a quadratic exension of $k^{(d=1)}$ as in assertion (i) and $\alpha'$ an automorphism of $G_{k'}$ which is not $(\bQ_{p_k})_+$-characteristic. Then $\phi(\alpha') \in {\rm Aut}(G_k)$ [cf.\ (i)] is not $(\bQ_{p_k})_+$-characteristic. \label{lemma:2.7:statement:2}
\end{enumerate}
\end{lem}
\begin{proof}
First, we verify assertion (i). Since the MLF $k$ is absolutely abelian, and $d_k$ is even, ${\rm Gal}(k/k^{(d=1)})$ is a finite abelian group of even order. Thus, it follows immediately from elementary group theory and Galois theory that there exists a quadratic extension $k'$ of $k^{(d=1)}$ contained in $k$. Next, we verify that $G_k$ is a characteristic subgroup of $G_{k'}$. Let $\beta$ be an automorphism of $G_{k'}$. Since $k$ is absolutely abelian, $k$ is Galois-specifiable [cf.\ \cite{Hoshi2}, Definition 6.1, and \cite{Hoshi2}, Theorem F, (i)]. Thus, it follows from Galois theory that there exists $\tau \in {\rm Gal}(\overline{k}/k^{(d=1)})$ such that $\beta(G_k)=\tau G_k \tau^{-1}$. Moreover, since $k$ is absolutely abelian, $G_k$ is a normal subgroup of ${\rm Gal}(\overline{k}/k^{(d=1)})$. In particular, we get $\beta(G_k) = \tau G_k \tau^{-1} = G_k$. This completes the proof of assertion (i).

Next, we verify assertion (ii). Let us first observe that it follows immediately from the various definitions involved that the diagram 
\[
\begin{tikzcd}
k^{(d=1)}_+ \arrow[r, hook] \arrow[d, "\left. \phi(\alpha')_+ \right|_{k^{(d=1)}_+}"'] & k'_+ \arrow[r, hook] \arrow[d, "\alpha'_+"'] & k_+ \arrow[d, "\phi(\alpha')_+"] \\
\alpha'_+(k^{(d=1)}_+) \arrow[r, hook]                                                 & k'_+ \arrow[r, hook]                         & k_+                             
\end{tikzcd}
\]
commutes, where the horizontal arrows are the natural inclusions, and we write $\alpha'_+$ (resp.\ $\phi(\alpha')_+$) for the group-theoretic automorphism induced by $\alpha' \in \Aut(G_{k'})$ (resp.\ $\phi(\alpha') \in \Aut(G_k)$). 
Since $\alpha'_+$ is not $(\bQ_{p_k})_+$-characteristic, $\alpha'_+(k_+^{(d=1)}) \neq k_+^{(d=1)}$. 
Thus, we conclude from the above diagram that $\phi(\alpha')_+$, hence also $\phi(\alpha')$, is not $(\bQ_{p_k})_+$-characteristic. This completes the proof of assertion (ii). 
\end{proof}

\begin{theorem}\label{one:theorem:2.8}
Let $k$ be an absolutely abelian MLF such that $p_k$ is odd, and $d_k$ is even. Then there exists an automorphism $\alpha \in {\rm Aut}(G_k)$ such that, for each nonzero integer $n$, $\alpha^n$ is not $(\bQ_{p_k})_+$-characteristic. 
\end{theorem}
\begin{proof}
This assertion follows from Lemma \ref{lemma:2.5} and Lemma \ref{old:lemma:2.7}, (i), (ii). 
\end{proof}


\section{The outer automorphism group of the absolute Galois group of an absolutely abelian MLF of even degree}\label{section3}

In the present \S \ref{section3}, we discuss the outer automorphism
group of the absolute Galois group of a certain MLF.  In the present
\S \ref{section3}, we maintain the notational conventions introduced
at the beginning of the preceding \S \ref{section2}.  Write, moreover, $\mathrm{Aut}(k)$ for the group of automorphisms of the field $k$ and $\mathrm{Out}(G_k)$ for the group of outer automorphisms of the group $G_k$.  Thus, we have a natural injective [cf.\ \cite{Hoshi1}, Proposition 2.1] homomorphism $\mathrm{Aut}(k) \hookrightarrow \mathrm{Out}(G_k)$ of groups.  In the present \S \ref{section3}, let us regard $\mathrm{Aut}(k)$ as a subgroup of $\mathrm{Out}(G_k)$:  
\[
\mathrm{Aut}(k) \subseteq \mathrm{Out}(G_k).  
\]

\begin{lem}\label{lemma:2.3}
Let $K$ be a field and $L$ a finite Galois extension of $K$ of extension degree invertible in $L$. Let $\alpha$ be an automorphism of the module $L_+$ which is compatible, relative to some automorphism of $\mathrm{Gal}(L/K)$ [which is not necessarily the identity automorphism], with the natural action of ${\rm Gal} (L/K)$ on $L_+$ and fits into the commutative diagram of modules 
\[
\begin{tikzcd}
L_{+} \arrow[rr, "{\rm Tr}_{L/K}"] \arrow[d, "\alpha"'] && K_+ \arrow[d, equal] \\
L_{+} \arrow[rr, "{\rm Tr}_{L/K}"']                         && K_+,                
\end{tikzcd}
\]
where we write ${\rm Tr}_{L/K}$ for the trace map with respect to the finite extension $L/K$. 
Then $\alpha$ restricts to the identity automorphism of the submodule $K_+ \subseteq L_+$.  
\end{lem}
\begin{proof}
Write $\beta \defeq \alpha - {\rm id} \in \End(L_+)$. Then it is immediate that the sequence 
\[
\begin{tikzcd}
0 \arrow[r] & \mbox{Ker}(\beta) \arrow[r] & L_{+} \arrow[r] & \mbox{Im}(\beta) \arrow[r] & 0, 
\end{tikzcd}
\]
hence also [cf.\ our assumption that $\alpha$ is compatible, relative to some automorphism of $\mathrm{Gal}(L/K)$, with the natural action of $\mathrm{Gal}(L / K)$] the sequence 
\[
\begin{tikzcd}
0 \arrow[r] & {\mbox{Ker}(\beta)}^{{\rm Gal}(L/K)} \arrow[r] & K_+ \arrow[r] & {\mbox{Im}(\beta)}^{{\rm Gal}(L/K)}, 
\end{tikzcd}
\]
is exact. 
Now observe that it follows from the commutative diagram in the statement of Lemma \ref{lemma:2.3} and the definition of $\beta$ that the image of $\Im(\beta)$ by $\mathrm{Tr}_{L / K}$ is zero. Thus, since ${\mbox{Im}(\beta)}^{{\rm Gal}(L/K)}$ is contained in $K_+$, and the degree of the finite extension $L / K$ is invertible in $L$, we conclude that ${\mbox{Im}(\beta)}^{{\rm Gal}(L/K)} = \{0\}$.  In particular, it follows from the above exact sequence that $\Ker(\beta)^{{\rm Gal}(L/K)} = K_{+}$, which implies that $\alpha(x) = x$ for each $x \in K_+$. This completes the proof of Lemma \ref{lemma:2.3}.  
\end{proof}

\begin{theorem}\label{theorem:2.4}
Let $k$ be an MLF and $\alpha$ an automorphism of $G_k$. Suppose that $d_k = 2$. Write $\alpha_+ \in \mathrm{Aut}(k_+)$ for the group-theoretic automorphism induced by $\alpha$. Then the following are equivalent: 
   \begin{enumerate}[label=(\arabic*),ref=(\arabic*)]
   	\item[\rm (1)] The automorphism $\alpha$ is $(\bQ_{p_k})_+$-preserving. \label{theorem:2.4:statement:1}
   	\item[\rm (2)] The automorphism $\alpha$ is $(\bQ_{p_k})_+$-characteristic. \label{theorem:2.4:statement:2}
   	\item[\rm (3)] The automorphism $\alpha_{+}$ is compatible with the natural action of ${\rm Gal} (k/k^{(d=1)})$ on $k_{+}$. \label{theorem:2.4:statement:3}
   \end{enumerate}

\end{theorem}

\begin{proof}
First, (1)$\Longrightarrow$(2) is immediate. Next, we verify (2)$\Longrightarrow$(3). 
\begin{comment}
We can write $k=k^{(d=1)}(\sqrt{a})$ for some a $\in k^{(d=1)}$. Since assertion (ii) holds and $\alpha_{+}$ is a isomorphism of $\bQ_{p_k}$-vector spaces, $\alpha_{+}(x+y \sqrt{a}) = x + y'\sqrt{a}$ for some $y' \in k^{(d=1)}$. Thus, $\sigma( \alpha_{+}(x+y \sqrt{a}) ) = \sigma(x + y'\sqrt{a} ) = x - y'\sqrt{a} = \alpha(\sigma(x+y\sqrt{a}))$. This completes the proof of (2)$\Longrightarrow$(3). 
\end{comment}
Suppose that (2) is satisfied.  Let us first observe that one may write $k = k^{(d=1)}(\sqrt{a})$ for some $a \in k^{(d=1)}$.  Since (2) is satisfied, $\alpha_+(1) = b$ for some $b \in k^{(d=1)}$.  Moreover, it follows from the final portion of Lemma \ref{lemma:2.2:statement:1}, (ii), that $\alpha_+(\sqrt{a}) = c \sqrt{a}$ for some $c \in k^{(d=1)}$. Thus, since $\alpha_+$ is an automorphism of $\bQ_{p_k}$-vector space, it follows that, for arbitrary $x$, $y \in k^{(d=1)}$, the equalities 
\[
\sigma(\alpha_{+}(x+y \sqrt{a})) = \sigma(bx + cy\sqrt{a}) =  \alpha_+(\sigma(x+y\sqrt{a})) \ \ \ (\sigma \in {\rm Gal}(k/k^{(d=1)}))
\]
hold. This completes the proof of (2)$\Longrightarrow$(3). 
Finally, (3)$\Longrightarrow$(1) follows immediately from Lemma \ref{lemma:2.2:statement:2}, (ii), and Lemma \ref{lemma:2.3}. 
\end{proof}

\begin{lem}\label{new:lemma:2.7}
Let $\alpha$ be an automorphism of $G_k$. Suppose that $k$ is absolutely Galois. 
If the image of $\alpha$ in $\mathrm{Out}(G_k)$ is contained in ${\rm N}_{{\rm Out}(G_k)}({\rm Aut}(k))$, then $\alpha$ is $(\bQ_{p_k})_+$-preserving. 
\end{lem}
\begin{proof}
Suppose that the image of $\alpha$ in ${\rm Out}(G_k)$ is contained in ${\rm N}_{{\rm Out}(G_k)}({\rm Aut}(k))$. 
Thus, $\alpha_+$ is compatible, relative to some automorphism of $\mathrm{Gal}(k/k^{(d=1)})$ [which is not necessarily the identity automorphism], with the natural action of ${\rm Gal}(k/k^{(d=1)}) = \Aut(k)$ on $k_+$. 
In particular, it follows from Lemma~\ref{lemma:2.2}, (ii), and Lemma~\ref{lemma:2.3} that $\alpha$ is $(\bQ_{p_k})_+$-preserving.  
\end{proof}

\begin{theorem}\label{two:theorem:2.8}
Let $k$ be an absolutely abelian MLF such that $p_k$ is odd, and $d_k$ is even. Then the set of ${\rm Out}(G_k)$-conjugates of the subgroup ${\rm Aut}(k) \subseteq {\rm Out}(G_k)$ is infinite. 
\end{theorem}

\begin{proof}
This assertion follows from Theorem ~\ref{one:theorem:2.8} and Lemma \ref{new:lemma:2.7}. 
\end{proof}

\begin{corollary}\label{corollary:2.9}
Let $k$ be an absolutely abelian MLF such that $p_k$ is odd, and $d_k$ is even. Then the following hold: 
   \begin{enumerate}[label=(\arabic*),ref=(\arabic*)]
   	\item[\rm (i)] The subgroup ${\rm Aut}(k)$ of ${\rm Out}(G_k)$ is not normal. \label{colloraly:2.9:statement:1}
   	\item[\rm (ii)] There exist infinitely many distinct [necesarily finite] subgroups of ${\rm Out}(G_k)$ isomorphic to ${\rm Aut}(k)$.  \label{corollary:2.9:statement:2}   
   \end{enumerate}
\end{corollary}
\begin{proof}
These assertions follow immediately from Theorem \ref{two:theorem:2.8}. 
\end{proof}

\begin{remark}\label{remark:2.10}
Let us recall from \cite{Hoshi2}, Theorem G, (iii), that 
if $p_k$ is odd, and $k$  is obtained by adjoining, to $k^{(d=1)}$, a 
primitive $p_k$-th root of unity and a $p_k$-th root of 
$p_k$, then the subgroup $\mathrm{Aut}(k) \subseteq 
\mathrm{Out}(G_k)$ is not normal. 
\end{remark}

\begin{remark}\label{remark:2.11}
The issue of whether or not a functorial group-theoretic 
reconstruction, from the group $G_k$, of the 
``field-theoretic'' subgroup $\mathrm{Aut}(k) \subseteq 
\mathrm{Out}(G_k)$ of the outer automorphism group 
$\mathrm{Out}(G_k)$ can be established is interesting 
from the point of view of the anabelian geometry of 
mixed-characteristic local fields.  Now let us recall 
that if the MLF $k$ is absolutely abelian, then, roughly 
speaking, one may reconstruct group-theoretically, from 
the group $G_k$, the set of $\mathrm{Out}(G_k)$-conjugates 
of the subgroup $\mathrm{Aut}(k) \subseteq 
\mathrm{Out}(G_k)$ [cf.\ \cite{Hoshi2}, Theorem F, (i), and 
\cite{Hoshi2}, Theorem 6.12, (ii)].  
On the other hand, Corollary \ref{corollary:2.9}, (i), implies that if 
$p_k$ is odd, $d_k$ is even, and $k$ is absolutely abelian, 
then, roughly speaking, it is impossible to establish a functorial group-theoretic reconstruction of the subgroup 
$\mathrm{Aut}(k) \subseteq \mathrm{Out}(G_k)$ itself [i.e., 
as opposed to the set of $\mathrm{Out}(G_k)$-conjugates of the 
subgroup $\mathrm{Aut}(k) \subseteq \mathrm{Out}(G_k)$]. 
\end{remark}

\begin{corollary}\label{Corollary 3.8} 
Let $k$ be an MLF such that $p_k$ is odd, and $d_k = 2$. 
Then the group-theoretic automorphism of $k_+$ induced by an 
automorphism of $G_k$ which lifts an element of the center of 
$\mathrm{Out}(G_k)$ is the identity automorphism of $k_+$.  
\end{corollary}
\begin{proof}
Let $\gamma$ be an element of the center of $\mathrm{Out}(G_k)$.  
Write $\gamma_+ \in \mathrm{Aut}(k_+)$ for the group-theoretic 
automorphism of $k_+$ induced by an automorphism of $G_k$ which 
lifts $\gamma$.  [Note that one verifies easily that $\gamma_+$ 
does not depend on the choice of such a lifting.]  Then it follows from Lemma \ref{new:lemma:2.7} that $\gamma_+$ is $(\mathbb{Q}_{p_k})_+$-preserving. 

Next, let $\alpha_+ \in \mathrm{Aut}(k_+)$ be a group-theoretic 
automorphism of $k_+$ which is not $(\mathbb{Q}_{p_k})_+$-characteristic 
[cf.\ Theorem \ref{one:theorem:2.8}]. Then since $\gamma$ is an element 
of the center of $\mathrm{Out}(G_k)$, one verifies immediately that 
$\gamma_+$ commutes with $\alpha_+$.  In particular, since $\gamma_+$ 
is $(\mathbb{Q}_{p_k})_+$-preserving, $\gamma_+$ restricts to the identity automorphism of $\alpha_+(k_+^{(d=1)}) \subseteq k_+$. 
Thus, since $d_k = 2$, and $k_+^{(d=1)} \neq \alpha_+(k_+^{(d=1)})$, 
we conclude that $\gamma_+$ is the identity automorphism of $k_+$, 
as desired.  This completes the proof of Corollary \ref{Corollary 3.8}. 
\end{proof}

\vspace{0.5mm}

\begin{acknowledgements}
The first author would like to thank Shinichi Mochizuki for a discussion related to the content of \S \ref{section1}. The first author was supported by JSPS KAKENHI Grant Number 18K03239. 
The second author would like to express a deepest gratitude to Hiroki Nishio and Keiko Nishio, for giving him constant support, warm encouragements. 
This research was supported by the Research Institute for Mathematical 
Sciences, an International Joint Usage/Research Center located in Kyoto 
University. 
\end{acknowledgements}        
        
\begin{thebibliography}{99}

\bibitem{Hoshi-b}
Y.\ Hoshi, 
A note on the geometricity of open homomorphisms between the absolute Galois groups of $p$-adic local fields, 
Kodai Math.\ J.\ 36 (2013), no.\ 2, 284--298.

\bibitem{Hoshi2}
Y.\ Hoshi, Topics in the anabelian geometry of mixed-characteristic local fields, Hiroshima Math.\ J.\ 49 (2019), no.\ 3, 323–398.

\bibitem{Hoshi4}
Y.\ Hoshi, Mono-anabelian reconstruction of number fields, RIMS K\^oky\^uroku Bessatsu B76 (2019), 1--77.

\bibitem{Hoshi1}
Y.\ Hoshi, Introduction to mono-anabelian geometry, to appear in Proceedings of the conference ``Fundamental Groups in Arithmetic Geometry'', Paris, France 2016.

\bibitem{Mzk}
S.\ Mochizuki, 
Topics in absolute anabelian geometry I:\ generalities, 
J.\ Math.\ Sci.\ Univ.\ Tokyo 19 (2012), no.\ 2, 139--242.  

\bibitem{NSW}
J.\ Neukirch, A.\ Schmidt, and K.\ Wingberg, Cohomology of number fields, Second edition.\
Grundlehren der Mathematischen Wissenschaften, 323.\ Springer-Verlag, Berlin, 2008.

\end{thebibliography}


\end{document}
